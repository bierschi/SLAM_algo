\chapter{Verwendetes Betriebssystem}
In diesem Kapitel wird die Installation des Betriebssystem Ubuntu Mate, sowie die Einrichtung des Frameworks ROS näher erläutert. Außerdem werden alle nötigen Konfigurationen gezeigt, um die Software kompilieren und ausführen zu können. 


\section{Installation von Ubuntu Mate auf Raspberry Pi 3b+}
Aufgrund einer sehr großen Ubuntu Community und die gute Anbindung an das Framework ROS, wurde sich für das Betriebssystem Ubuntu Mate entschieden. Da keine durchgängige Echtzeitanbindung gefordert ist, werden hier der SLAM sowie die Wegefindung berechnet und ausgeführt. Die Sensoren der Motorsteuerung agieren dagegen auf einem STM Board um schneller auf Änderungen reagieren zu können. 
Da zum aktuellen Zeitpunkt (25.10.2018) kein angepasstes Betriebssystem für den Raspberry Pi 3b+ zur Verfügung steht, mussten kleinere Konfigurationen stattfinden um das vorhandene Raspberry Pi 3 Image auf einem Rasperry Pi 3b+ zum laufen zu bekommen. Nachfolgend werden alle Konfigurationen genauer erläutert. 

1. Download eines Images für Raspberry  Pi 2/3 auf folgender Seite:

2. Flashen des Images auf einer SD-Karte mit Win32DiskImages oder (linux):

3. Diese SD-Karte in einen \textbf{Raspberry Pi 2} oder \textbf{Raspberry Pi 3} einstecken und starten. 

4. 






\section{Installation des Frameworks Robot Operating System (ROS)}

\section{Roscore Master und Launch file als Systemd Service}