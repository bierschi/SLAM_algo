\chapter{Zusammenfassung und Ausblick}


In diesem Projektstudium wurde ein verbesserter Aufbau eines autonomen Laser Fahrzeugs (ALF) durchgeführt. Da zu Beginn des Projektes die in Kapitel 2 erläuterten Probleme vorhanden waren, wurde dieser aufwendigere Weg gewählt und dieses Projekt in Hard- und Software komplett überarbeitet.\\
Bei der Hardware wurde sich für einen Raspberry Pi 3b+ als Hauptrechner entschieden. Auf diesem werden der SLAM-Algorithmus sowie die Wegberechnung durchgeführt. Außerdem sind noch genug Ressourcen übrig für die Entwicklung weiterer Module. Für die Ansteuerung der Sensorik wird das STM32 Board verwendet. Durch die Ausgliederung können die Sensoren in einer deterministischen Zeit ausgelesen und kontrolliert werden. Auch an diesem Board sind noch Kapazitäten offen für das Hinzufügen weiterer Sensoren.\\
Bei der Software wurde unter anderem Wert gelegt auf eine Projektstruktur, welche leicht erweiterbar ist. Hierbei wurde auf das plattformunabhängige Tool CMake gesetzt, mit dieser schon zahlreiche große Projekte erfolgreich umgesetzt werden konnte \cite{cmake.2018}. In diesem Projektstudium wurden vier Module realisiert (\textit{communication, motioncontrol, pathfinder, slam}). Dadurch ist es möglich, dass Alf einen Raum autonom erkundet und zeitgleich eine Karte der Umgebung aufbaut.

Als Ausblick für weitere Gruppen wäre es möglich, am aktuellen Stand weiterzuentwickeln und die autonome Erkundung des ALF noch weiter zu verbessern. Alternativ könnte auch ein Modul umgesetzt werden, welches die Lokalisierung in einer bereits aufgebauten Karte vornimmt. Man lädt zur Laufzeit eine vorhandene Karte und ALF muss versuchen sich in dieser sehr genau zu lokalisieren. Dabei könnte dann bestimmte Punkte ausgewählt werden, die der ALF dann anfahren muss.
