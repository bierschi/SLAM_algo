

\chapter{Example for a table and a listing}



\subsection{Table with subtitle}

\begin{table}[h!]
\centering
\begin{tabular}{||c c c c||} 
 \hline
 Col1 & Col2 & Col2 & Col3 \\ [0.5ex] 
 \hline\hline
 1 & 6 & 87837 & 787 \\ 
 2 & 7 & 78 & 5415 \\
 3 & 545 & 778 & 7507 \\
 4 & 545 & 18744 & 7560 \\
 5 & 88 & 788 & 6344 \\ [1ex] 
 \hline
\end{tabular}
\caption{Table to test captions and labels}
\label{table:1}
\end{table}



%\newpage
\section{Show code nicely}

Here´s an example of how to correctly render program code:

\begin{figure}[h]
\begin{lstlisting}[
backgroundcolor={\color{lightgray}},
basicstyle={\normalsize\sffamily},
breaklines=true,
frame={leftline,bottomline,rightline,topline},
language=C,
numbers=left,
showstringspaces=false,
xleftmargin=15pt
]
#include <stdio.h> 


main() 
	{
        int i;
        for (i = 0; i<=2; i++)
        {
        fork();
        setvbuf(stdout, NULL, _IOLBF, 0); //(Stream, Pufferangabe, Puffertyp, Puffergroesse)
        printf("PID=%6d, i=%d\n", getpid(), i);
        }
} 
\end{lstlisting}


\protect\caption{Modified program with setvbuf()}
\end{figure}

