
\vspace{15.0mm}


\addchap*{Abstract}


In diesem Projektstudium wurde ein verbesserter Aufbau eines autonomen Laser Fahrzeugs (ALF) durchgeführt. Das Ziel hierbei war es, das Fahrzeug autonom einen Raum erkunden zu lassen und zeitgleich eine Umgebungskarte aufzubauen.\\
Bei der Hardware wurde als Hauptrechner ein Raspberry Pi 3b+ verwendet. Dieser besitzt genug Ressourcen um noch weitere Module zu entwickeln. Außerdem wurde für die Steuerung der Sensorik ein STM32 Board eingesetzt um diese in einer deterministischen Zeit auslesen und kontrollieren zu können.\\
Die Software wurde in C++ entwickelt und die Projektstruktur mit dem Tool CMake aufgebaut. Somit wird sichergestellt, dass auch weitere Gruppen sich schnell einarbeiten und neue Module entwickeln können. Zum aktuellen Zeitpunkt wurden 3 aktive Module und 2 passive Module umgesetzt. Zu den aktiven Modulen gehören eine Bewegungssteuerung als \textit{motioncontrol}, eine Wegfindung als \textit{pathfinder} und ein SLAM-Algorithmus als \textit{slam} (sind aktuell im Einsatz). Zu den beiden passiven Modulen gehören eine Server/Client Verbindung als \textit{communication} und eine Lidar Verbindung als \textit{lidar\_connect}, welche aktuell nicht eingesetzt werden. Diese können aber von weiteren Gruppen zusätzlich verwendet werden.  
