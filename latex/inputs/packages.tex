
%\renewcommand{\familydefault}{\sfdefault}
\renewcommand*\familydefault{\rmdefault}		% family font


\usepackage[margin=0.7in]{geometry}				% geometry settings in document
\usepackage[utf8]{inputenc}						% encoding
\usepackage{amsmath,amssymb,amsfonts,amsthm}	% related to math

\usepackage{multirow}  							% statement \multirow
\usepackage{graphicx}  							% statement \includegraphics
\usepackage{setspace}							% statement \begin{doublespace}
\usepackage{varioref}
%\usepackage[
%	automark,								 	% chapter in headline
%	headsepline,								% dividing line under header
%	ilines										% align the dividing line to the left
%]{scrpage2}

\usepackage[acronym,toc,automake]{glossaries}   % list of abbreviation, toc
\usepackage{scrlayer-scrpage}
\usepackage{color, xcolor}						% change the color of the font
\usepackage{calc}								% calculation

% show all cross-references and URLs as a link
\usepackage[
colorlinks=true, bookmarks, bookmarksnumbered, bookmarksopen, bookmarksopenlevel=1,
linkcolor=black, citecolor=black, urlcolor=blue, filecolor=blue,
pdfpagelayout=OneColumn, pdfnewwindow=true,
pdfstartview=XYZ, plainpages=false, pdfpagelabels,
pdfauthor={LyX Team}, pdftex,
pdftitle={LyX's Figure, Table, Floats, Notes, and Boxes manual},
pdfsubject={LyX-documentation about figures, tables, floats, notes, and boxes},
pdfkeywords={LyX, Tables, Figures, Floats, Boxes, Notes}]{hyperref}

\usepackage{textcomp,units}

\usepackage{booktabs}

\usepackage{listings}

\usepackage[german]{babel}

\definecolor{mygreen}{rgb}{0,0.6,0}
\definecolor{mygray}{rgb}{0.83,0.83,0.83}
\definecolor{mymauve}{rgb}{0.58,0,0.82}

\lstset{ 
  backgroundcolor=\color{mygray},   % choose the background color; you must add \usepackage{color} or 
  basicstyle=\footnotesize,        % the size of the fonts that are used for the code
  breakatwhitespace=false,         % sets if automatic breaks should only happen at whitespace
  breaklines=true,                 % sets automatic line breaking
  captionpos=b,                    % sets the caption-position to bottom
  commentstyle=\color{mygreen},    % comment style
  deletekeywords={...},            % if you want to delete keywords from the given language
  escapeinside={\%*}{*)},          % if you want to add LaTeX within your code
  extendedchars=true,              % lets you use non-ASCII characters; for 8-bits encodings only, does not work with UTF-8
  frame=single,	                   % adds a frame around the code
  keepspaces=true,                 % keeps spaces in text, useful for keeping indentation of code (possibly needs columns=flexible)
  keywordstyle=\color{blue},       % keyword style
  language=bash,                 % the language of the code
  morekeywords={sudo, cp, wget, apt-get, mkdir},            % if you want to add more keywords to the set
  numbers=left,                    % where to put the line-numbers; possible values are (none, left, right)
  numbersep=5pt,                   % how far the line-numbers are from the code
  numberstyle=\tiny\color{mygray}, % the style that is used for the line-numbers
  rulecolor=\color{black},         % if not set, the frame-color may be changed on line-breaks within not-black text (e.g. comments (green here))
  showspaces=false,                % show spaces everywhere adding particular underscores; it overrides 'showstringspaces'
  showstringspaces=false,          % underline spaces within strings only
  showtabs=false,                  % show tabs within strings adding particular underscores
  stepnumber=2,                    % the step between two line-numbers. If it's 1, each line will be numbered
  stringstyle=\color{mymauve},     % string literal style
  tabsize=2,	                   % sets default tabsize to 2 spaces
  title=\lstname                   % show the filename of files included with \lstinputlisting; also try caption instead of title
}